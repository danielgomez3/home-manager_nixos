\documentclass[12pt]{article}

%
%Margin - 1 inch on all sides
%
\usepackage[letterpaper]{geometry}
\usepackage{times}
\geometry{top=1.0in, bottom=1.0in, left=1.0in, right=1.0in}

%
%Doublespacing
%
\usepackage{setspace}
\doublespacing

%
%Rotating tables (e.g. sideways when too long)
%
\usepackage{rotating}


%
%Fancy-header package to modify header/page numbering (insert last name)
%
\usepackage{fancyhdr}
\pagestyle{fancy}
\lhead{} 
\chead{} 
\rhead{Gomez \thepage} 
\lfoot{} 
\cfoot{} 
\rfoot{} 
\renewcommand{\headrulewidth}{0pt} 
\renewcommand{\footrulewidth}{0pt} 
%To make sure we actually have header 0.5in away from top edge
%12pt is one-sixth of an inch. Subtract this from 0.5in to get headsep value
\setlength\headsep{0.333in}


\usepackage[smartEllipses]{markdown}


%
%Works cited environment
%(to start, use \begin{workscited...}, each entry preceded by \bibent)
% - from Ryan Alcock's MLA style file
%
\newcommand{\bibent}{\noindent \hangindent 40pt}
\newenvironment{workscited}{\newpage \begin{center} Works Cited \end{center}}{\newpage }


%
%Begin document
%
\begin{document}
\begin{flushleft}

%%%%First page name, class, etc
Daniel Gomez\\
Jim Richards\\
Introduction to Literature (eng151)\\
May 2 2023\\


%%%%Title
\begin{center}
The Non-fictional benefits of Fiction
\end{center}


%%%%Changes paragraph indentation to 0.5in
\setlength{\parindent}{0.5in}
%%%%Begin body of paper here

Fiction implies stories, and lies. Stories and metaphors can be lies in the direct sense; they never happened (at least to your knowledge), or otherwise not capable of fruition in our reality. 

Sadly, since the majority of those reading are adults, we understand and accept nuance. This mean we're not able to write off fiction entirely because of this reasoning. Just because something hasn't happened in exactness in real life, doesn't mean that stories are complete lies. In fact, it's important to realize that they more than often do contain truth, but not in the traditional sense. Stories offer truth in a much more useful way than fiction. How could this be? 

Here's a fact: Humans aren't robots. And? People don't absorb information plain information well like computers are meant to: people are actually much better at something more directly useful than remembering things. People come from a long evolutionary tree of pattern-recognition.

There has been extensive study on humans and their ability to recall information based on experience, memory 'paths', and repetition. Fiction perfectly occupies this niche through plot, symbolism, and allegory. 

In fact, some of the oldest writing ever recorded, including and namely 'the Epic of Gilgamesh', is an epic poem based in fiction written from ancient Mesopotamia.
The purpose of which was to teach people about the human condition, and the virtue of pursuing what's spiritual by forsaking what's carnal.

It does it successfully. None of the events were real. Again, how could this be? It's because stories communicate idea better than plain information. Drama captivates. Exaggeration lures.

Children, with less of a capability in understanding complex nuance and with a growing hippocampus, struggle to memorize or find memorization for its sake, when contrasted with literary tools used by Fiction.

We crave stories because we often and historically long for escapism from time to time. We know that healthy escapism-- whether in the form of literature, sports, or games-- can stimulate inspiration and trigger a relaxed response.


Joseph Smith in his life gave an allegory to the early missionary and friend Robert B. Thompson, warning him of the detriment of never relaxing or putting one's mind and body at ease.

“‘Robert’, Joseph said, ‘you have been so faithful and relentless in this work, you need to relax, go out, have a spree, relax.’ But Thompson was a sober man and said, ‘I can’t do it.’ Joseph said, ‘You must do it, or if you don’t do it, you will die.'” (Joseph Smith Papers).

The fictional allegory Joseph Smith later used compared a man to a bow that was constantly 'strung tight' would soon lose its spring.
"The bow must be unstrung." (Joseph Smith Papers).

In a nonfiction way, commanding Robert to relax and verbosely listing every detail of how to relax, why, and what he was doing wrong would only serve him wrong: It's not likely he would remember every detail. This is human nature. It was always much more effective to instruct using analogy and allegory to draw parallel and strike pattern-recognition.

The short story "Who's Irish?" by Gish Jen is another great example of this.
This work of Fiction tells a story of a Chinese grandmother forced to live in an environment she's not comfortable in, versus where she initially hailed.
On top of this, she has a spoiled and misbehaving grandchild otherwise begging to be physically disciplined.

Again, although this writing might be in direct contrast to a specific name, area, or timeline, its existence in the general sense as not at all questioned. These events is more than feasible and have in fact occur all the time.
Her struggles are not unheard, and people realate to this fiction. Not only that, we gain insight and deal with real-world issues that come with cultural clash.



Fiction is more than story. It's instructive, inspiring, and informative. The purpose varies greatly, and this is not something as easily afforded and expressed in Non-Fictional work.





\newpage


%%%%%'Notes' for footnotes
%\begin{center}
%Notes
%\end{center}
%
%
%\setlength{\parindent}{0.5in}
%
%1. Danhof includes “Delaware, Maryland, all states north of the Potomac and Ohio rivers, Missouri, and states to its north” when referring to the northern states (11).
%
%
%2. For the purposes of this paper,“science” is defined as it was in nineteenthcentury agriculture: conducting experiments and engaging in research.
%
%
%3. Please note that any direct quotes from the nineteenth century texts are writtenin their original form, which may contain grammar mistakes according to twenty-first century grammar rules.

%%%%Works cited
\begin{workscited}

%%% \bibent
%%% Stevie Smith \textit{Not Waving but Drowning} 
%%% https://www.poetryfoundation.org/poems/46479/not-waving-but-drowning

\bibent
Joseph Smith \textit{The Joseph Smith Papers}
https://www.josephsmithpapers.org/person/robert-blashel-thompson

Gish Jen \textit{Who is Irish}
https://www.sjsu.edu/people/julie.sparks/courses/Engl1bspr2016/Gish%20Jen%20story%20Who%20is%20Irish.pdf

\end{workscited}

\end{flushleft}
\end{document}
\}