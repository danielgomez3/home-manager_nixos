\markdownRendererDocumentBegin
Dylan Thomas's poem "Do not go gentle into that good night" departs from poetic stereotype, and does not forfeit its right to poetry, nor in doing so seems contrived in any way.\markdownRendererInterblockSeparator
{}This poem is a classic example of a villanelle using Iambic pentameter. Conducive to the high structured stereotype of a villanelle, we see Thomas's poem containing exactly 19 lines in total, five tercets, and in the last stanza, a quatrain. \markdownRendererInterblockSeparator
{}Unless an analyst or poetic veteran were to take out a scratch paper, this poem would seem a faithful Iambic pentameter, but it is technically not. In the third Stanza, the second line stating "..Blind eyes could blaze like meteors and be gay,.." (Thomas, line 14) contains 11 Iambs, contrary to the needed 10 to qualify itself as an Iambic Pentameter. Would it therefore forfeit the right to itself as an Iambic pentameter? Perhaps; it certainly begs the question. Would this then effectively disqualify "Do not go gentle into that good night" as a poem altogether?\markdownRendererInterblockSeparator
{}First, let's discuss what departing from the order does for this poem. By rejecting what's usual of an Iambic pentameter, the poem immediately inherits uniqueness by virtue of deviation. But this shouldn't be necessary for every poem to be allowed uniqeness, neither should it be sought out ritually. This poem's rebellion does not feel contrived in any way.\markdownRendererInterblockSeparator
{}There is argument to be had that departure in this case was in fact purposeful. If the higher law of art was to be an ultimate form of expression, why should it be completely confined? If every poem existed entirely to faithful rule-abiding subset, we would soon run out of ideas and possible combinations of words (Unless we were to challenge ourselves with completely newer languages to write, but this would be outside the scope of possible and modern solutions-- we have not yet arrived at that necessity). We know that Art is not about expressing permutations for its sake.\markdownRendererInterblockSeparator
{}11 Iambs also puts emphasis on the line as a sort of poetic 'shock value', perhaps guaranteeing the attention to the beauty the verse does have to offer. Nothing else in this poem was so literarily and visually extreme as to depict the celestial and the debilitating, specifically '"blind[ness]" and the "meteors" respectively. It almost forces the reader to forget the objective entirely and mentally visualize in awe of these happening in real time.\markdownRendererInterblockSeparator
{}Thomas would in fact have never been able to combine those words together with the Iambically bloated word 'meteor' without breaking the rules. A question: what's more important than following the rules? It's expressing your feelings, and in doing so better obeying the higher law of poetry and art. \markdownRendererInterblockSeparator
{}Dylan Thomas's poem may no longer be an Iambic Pentameter because of his deliberate choice, but it has so much more to gain. It is a much more cohesive and purposeful poem because of it. Departing from themes and rituals is at the core of art: the dance between choas and order.\markdownRendererDocumentEnd