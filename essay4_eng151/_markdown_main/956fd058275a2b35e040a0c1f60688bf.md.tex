\markdownRendererDocumentBegin
The poem "Not Waving but Drowning" alludes to a theme that supersedes a character's tragic death. A story of death, although immediately shocking and speculative, I argue is not the most important thing that we can derive from this poem.\markdownRendererInterblockSeparator
{}"Not Waving but Drowning" hosts an 'ABCB' rhyme scheme, which qualifies the poem as a Quatrain, even when the poem's rhymes aren't readily audible and rely more on the reader's emphasis on poetic feet. We learn of someone, a male, who tells the account of his death postmortem. British author and poetess Stevie Smith in an interview explained the inspiration of her writing from a story she read of a man who was thought to be waving at his friend while attempting to swim in the sea, while in reality drowning.\markdownRendererInterblockSeparator
{}This poem inherits a lot from that story, while adding personality and more reasoning as to why someone might not be believable when struggling; maybe a sort of 'boy who cried wolf' scenario. This is evident in the beginning of the second stanza, mentioning how the "poor chap.." had "..always loved larking" (Smith) when living.\markdownRendererInterblockSeparator
{}But this still has no part in my argument. We are yet looking at a surface-level understanding when we continue to observe, assume, and speculate the death of the man (and perhaps this is our human nature to do so). We continue to lose out on even more when we investigate the origins/inspiration of the poem. It is only done so in my essay to save the reader any trouble in barefaced research, so we may speculate together with more confidence.\markdownRendererInterblockSeparator
{}This poem answers what it's like to be in a depressive or an otherwise emotionally helpless state. This poem explores a darker them: 'What is it like to be in an abusive, constraining, or self-deprecating relationship within one's own mind?', a fate often worse than immediate drowning. A fate that involves continual drowning, without a way to explain or the means by which one may call for help.\markdownRendererInterblockSeparator
{}If every line in this poem was purposeful, and assumed to have been crafted carefully with revision, we have reason to believe or assume that in the version we read the author made it clear to never mention the likes of any water. Yet when we read, we might assume a lake or ocean.\markdownRendererInterblockSeparator
{}Even in the poem, at the end of the second stanza: "It must have been too cold for him his heart gave way" (Smith), it still technically gives no indication of water whatsoever. The following stanza proves this by assuring "it was too cold always". If we were ready to assume that the one dead would have drowned solely by water, then by virtue we would then have to assume that he may as well have lived underwater.\markdownRendererInterblockSeparator
{}This Poem is not much about water and neither about underwater drowning. This is about feeling distant, being "..much too far out.." (Smith) all of one's life. Not by purpose, but by helplessness; a common rhetoric for a depressive state of being.\markdownRendererDocumentEnd